\documentclass[12pt,a4paper]{article}
\usepackage{ctex}
\usepackage{amsmath,amscd,amsbsy,amssymb,latexsym,url,bm,amsthm}
\usepackage{epsfig,graphicx,subfigure}
\usepackage{enumitem,balance}
\usepackage{wrapfig}
\usepackage{mathrsfs,euscript}
\usepackage[usenames]{xcolor}
\usepackage{hyperref}
\usepackage[vlined,ruled,linesnumbered]{algorithm2e}
\hypersetup{colorlinks=true,linkcolor=black}

\newtheorem{theorem}{Theorem}
\newtheorem{lemma}[theorem]{Lemma}
\newtheorem{proposition}[theorem]{Proposition}
\newtheorem{corollary}[theorem]{Corollary}
\newtheorem{exercise}{Exercise}
\newtheorem*{solution}{Solution}
\newtheorem{definition}{Definition}
\theoremstyle{definition}

\renewcommand{\thefootnote}{\fnsymbol{footnote}}

\newcommand{\postscript}[2]
 {\setlength{\epsfxsize}{#2\hsize}
  \centerline{\epsfbox{#1}}}

\renewcommand{\baselinestretch}{1.0}

\setlength{\oddsidemargin}{-0.365in}
\setlength{\evensidemargin}{-0.365in}
\setlength{\topmargin}{-0.3in}
\setlength{\headheight}{0in}
\setlength{\headsep}{0in}
\setlength{\textheight}{10.1in}
\setlength{\textwidth}{7in}
\makeatletter \renewenvironment{proof}[1][Proof] {\par\pushQED{\qed}\normalfont\topsep6\p@\@plus6\p@\relax\trivlist\item[\hskip\labelsep\bfseries#1\@addpunct{.}]\ignorespaces}{\popQED\endtrivlist\@endpefalse} \makeatother
\makeatletter
\renewenvironment{solution}[1][Solution] {\par\pushQED{\qed}\normalfont\topsep6\p@\@plus6\p@\relax\trivlist\item[\hskip\labelsep\bfseries#1\@addpunct{.}]\ignorespaces}{\popQED\endtrivlist\@endpefalse} \makeatother

\begin{document}
\noindent

%========================================================================
\noindent\framebox[\linewidth]{\shortstack[c]{
\Large{\textbf{Lab00-Proof}}\vspace{1mm}\\
CS214-Algorithm and Complexity, Xiaofeng Gao, Spring 2019.}}
\begin{center}
\footnotesize{\color{red}$*$ If there is any problem, please contact TA Jiahao Fan.}

% Please write down your name, student id and email.
\footnotesize{\color{blue}$*$ Name: Kylin Chen  \quad Student ID:517030910155 \quad Email: k1017856853@icloud.com}
\end{center}

\begin{enumerate}
    \item
    Prove that for any integer $n>2$, there is a prime $p$ satisfying $n<p<n!$. {\color{blue}(Hint: consider a prime factor $p$ of $n!-1$ and prove by contradiction)}
    \begin{proof}
        As for $n!-1$, it follows that $n<n!-1<n!$.\item
        If $n!-1$ is a prime number, the original statement is true.\item
        If $n!-1$ is not a prime number, we can suppose p is a prime number of $n!-1$.
        Then, we assume $p \le n$, so p can be a factor of $n!$, but $n!$ and $n!-1$ don't have any matual factor more than $1$. The contradiction negate the assumption that 
        $p \le n$, so $n<p<n!$.\item
        Above all, the original statement is true.
    \end{proof}

    \item
    Use the minimal counterexample principle to prove that for any integer $n>17$, there exist integers $i_n\ge 0$ and $j_n\ge 0$, such that $n = i_n \times 4 + j_n \times 7$.
    \begin{proof}
        If the statement is not true for every $n>17$, there must be a smallest such false value, say $n=k$.\item
        Since n=18, $n=1 \times 4 +2 \times 7$,which $i_0=1$ and $j_0=2$.\item
        Since n=k is the smallest false one, so $k-1=i_k \times 4 +j_k \times 7$, in which $i_k \ge i_0=1$ and $j_k \ge j_0=2$.\item
        However, $k=(k-1)+1=(i_k+1) \times 4 +(j_k-1) \times 7$, so we can derived a contradiction,which allows us to conclude that our original assumption is false.
    \end{proof}

    \item
    Suppose $a_0=1$, $a_1=2$, $a_2=3$, and $a_k=a_{k-1}+a_{k-2}+a_{k-3}$ for $k \ge 3$. Use the strong principle of mathematical induction to prove that $a_n \le 2^n$ for any integer $n\ge 0$.
    \begin{proof}
        As for k=3, $a_3=a_0+a_1+a_2=6 < 2^3 = 8$,the original statement is true.\item
        We can assume that for $3\le k\le n-1$, the statement is true.\item
        As for k=n, $a_n=a_{n-1}+a_{n-2}+a_{n-3}\le 2^{n-1}+2^{n-2}+2^{n-3}=7\times 2^{n-3}<2^n$ 
        Above all, the original statement is true.
    \end{proof}

    \item
    Prove, by mathematical induction, that
    $$
    (n+1)^2+(n+2)^2+(n+3)^2+\cdots +(2n)^2=\dfrac{n(2n+1)(7n+1)}{6}
    $$
    is true for any integer $n\ge 1$.
    \begin{proof}
        For $n=1$, the equation clearly holds.\item
        Suppose the equation holds when $n=k$, it means $(k+1)^2+(k+2)^2+(k+3)^2+\cdots +(2k)^2=\dfrac{k(2k+1)(7k+1)}{6}$. \item
        As for $n=k+1$, $(k+2)^2+(k+3)^2+(k+4)^2+\cdots +(2(k+1))^2$\item$=\dfrac{k(2k+1)(7k+1)}{6}+(2k+1)^2+(2k+2)^2-(k+1)^2$\item$=\dfrac{(k+1)(2k+3)(7k+8)}{6}$.\item
        Above, for any integer $n\ge 1$, the original statement is true.
    \end{proof}

\end{enumerate}

\vspace{20pt}

\textbf{Remark:} You need to include your .pdf and .tex files in your uploaded .rar or .zip file.

%========================================================================
\end{document}
